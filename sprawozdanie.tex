\documentclass[12pt,a4paper]{article}
\usepackage[utf8]{inputenc}
\usepackage[polish]{babel}
\usepackage[T1]{fontenc}
\usepackage{graphicx}
\usepackage{amsmath}
\usepackage{booktabs}
\usepackage{float}
\usepackage{geometry}
\usepackage{listings}
\usepackage{xcolor}
\usepackage{hyperref}

\geometry{margin=2.5cm}

\lstset{
    basicstyle=\ttfamily\small,
    breaklines=true,
    frame=single,
    numbers=left,
    numberstyle=\tiny,
    captionpos=b
}

\title{\textbf{Sprawozdanie z projektu} \\
\large System Producent-Konsument z Tablicą Dispatcher}
\author{Marcel}
\date{\today}

\begin{document}

\maketitle
\tableofcontents
\newpage

\section{Wprowadzenie}

Celem projektu było zaprojektowanie i zaimplementowanie wielowątkowego systemu producent-konsument wykorzystującego mechanizm tablicy dispatcher. System umożliwia równoczesną pracę wielu producentów i konsumentów komunikujących się poprzez tablicę buforów FIFO.

\subsection{Założenia projektu}
\begin{itemize}
    \item Wielowątkowa implementacja w języku Java
    \item Synchronizacja dostępu do współdzielonych zasobów
    \item Równomierne rozłożenie obciążenia między bufory
    \item Możliwość konfiguracji liczby producentów, buforów i konsumentów
    \item Pomiar wydajności i analiza statystyczna
\end{itemize}

\section{Architektura systemu}

\subsection{Komponenty systemu}

System składa się z następujących elementów:

\begin{enumerate}
    \item \textbf{TablicaDispatcher} - centralny komponent zarządzający tablicą buforów i koordynujący komunikację
    \item \textbf{Producer} - wątki produkujące elementy i umieszczające je w buforach
    \item \textbf{Consumer} - wątki pobierające i przetwarzające elementy z buforów
    \item \textbf{TestTablicaDispatcher} - moduł testowy umożliwiający konfigurację i uruchomienie systemu
\end{enumerate}

\subsection{Mechanizm synchronizacji}

Do synchronizacji dostępu do współdzielonych zasobów wykorzystano:
\begin{itemize}
    \item \texttt{AtomicBoolean} - flaga zakończenia pracy systemu
    \item \texttt{AtomicInteger} - liczniki produkcji, konsumpcji i statystyk buforów
    \item Wbudowane mechanizmy synchronizacji kolejek \texttt{BlockingQueue}
\end{itemize}

\section{Konfiguracja testów}

Przeprowadzono trzy testy o różnych konfiguracjach:

\begin{table}[H]
\centering
\caption{Konfiguracje testowe}
\begin{tabular}{@{}ccccc@{}}
\toprule
\textbf{Test} & \textbf{Producenci (P)} & \textbf{Bufory (B)} & \textbf{Konsumenci (K)} & \textbf{Czas [s]} \\ \midrule
1 & 10 & 5 & 10 & 5 \\
2 & 100 & 5 & 100 & 5 \\
3 & 20 & 10 & 40 & 10 \\ \bottomrule
\end{tabular}
\end{table}

\section{Wyniki eksperymentów}

\subsection{Test 1: Mały system (10/5/10)}

\begin{table}[H]
\centering
\caption{Wyniki testu 1}
\begin{tabular}{@{}ll@{}}
\toprule
\textbf{Parametr} & \textbf{Wartość} \\ \midrule
Czas rzeczywisty & 5030 ms \\
Wyprodukowano elementów & 778~872 \\
Skonsumowano elementów & 776~923 \\
Przepustowość & 154~458 elem/s \\
Średnia na bufor & 155~384,6 \\
Odchylenie standardowe & 489,6 \\
Współczynnik zmienności & 0,3\% \\ \bottomrule
\end{tabular}
\end{table}

\textbf{Analiza:} System wykazał doskonałe równoważenie obciążenia z bardzo niskim współczynnikiem zmienności (0,3\%). Wszystkie bufory obsłużyły zbliżoną liczbę elementów (ok. 20\% każdy).

\subsection{Test 2: Duży system (100/5/100)}

\begin{table}[H]
\centering
\caption{Wyniki testu 2}
\begin{tabular}{@{}ll@{}}
\toprule
\textbf{Parametr} & \textbf{Wartość} \\ \midrule
Czas rzeczywisty & 5054 ms \\
Wyprodukowano elementów & 2~611~174 \\
Skonsumowano elementów & 2~611~093 \\
Przepustowość & 516~639 elem/s \\
Średnia na bufor & 522~218,6 \\
Odchylenie standardowe & 783,6 \\
Współczynnik zmienności & 0,2\% \\ \bottomrule
\end{tabular}
\end{table}

\textbf{Analiza:} Przy zwiększeniu liczby producentów i konsumentów do 100, system osiągnął 3,3-krotnie wyższą przepustowość (516~639 elem/s) przy zachowaniu doskonałego równoważenia (0,2\%).

\subsection{Test 3: Asymetryczny system (20/10/40)}

\begin{table}[H]
\centering
\caption{Wyniki testu 3}
\begin{tabular}{@{}ll@{}}
\toprule
\textbf{Parametr} & \textbf{Wartość} \\ \midrule
Czas rzeczywisty & 10~052 ms \\
Wyprodukowano elementów & 4~462~331 \\
Skonsumowano elementów & 4~461~549 \\
Przepustowość & 443~847 elem/s \\
Średnia na bufor & 446~154,9 \\
Odchylenie standardowe & 610,8 \\
Współczynnik zmienności & 0,1\% \\ \bottomrule
\end{tabular}
\end{table}

\textbf{Analiza:} System z większą liczbą konsumentów niż producentów oraz podwojoną liczbą buforów osiągnął najlepsze równoważenie (0,1\%). Każdy z 10 buforów obsłużył dokładnie 10\% elementów.

\section{Analiza porównawcza}

\subsection{Przepustowość systemu}

\begin{table}[H]
\centering
\caption{Porównanie przepustowości}
\begin{tabular}{@{}lccc@{}}
\toprule
\textbf{Test} & \textbf{Przepustowość [elem/s]} & \textbf{Wzrost} & \textbf{P/K ratio} \\ \midrule
Test 1 & 154~458 & - & 1,0 \\
Test 2 & 516~639 & +234\% & 1,0 \\
Test 3 & 443~847 & +187\% & 0,5 \\ \bottomrule
\end{tabular}
\end{table}

Najwyższą przepustowość osiągnięto w teście 2 ze zrównoważoną liczbą producentów i konsumentów (100/100).

\subsection{Równomierność obciążenia}

\begin{table}[H]
\centering
\caption{Współczynniki zmienności}
\begin{tabular}{@{}lccl@{}}
\toprule
\textbf{Test} & \textbf{Współczynnik [\%]} & \textbf{Liczba buforów} & \textbf{Ocena} \\ \midrule
Test 1 & 0,3 & 5 & Doskonała \\
Test 2 & 0,2 & 5 & Doskonała \\
Test 3 & 0,1 & 10 & Doskonała \\ \bottomrule
\end{tabular}
\end{table}

Wszystkie testy wykazały doskonałe równoważenie obciążenia (współczynnik < 10\%). Najlepszy wynik uzyskano w teście 3 z większą liczbą buforów.

\section{Wnioski}

\begin{enumerate}
    \item \textbf{Skalowalność} - System efektywnie skaluje się wraz ze wzrostem liczby wątków. Test 2 (100 producentów/konsumentów) osiągnął ponad 3-krotnie wyższą przepustowość niż test 1.
    
    \item \textbf{Równoważenie obciążenia} - Mechanizm tablicy dispatcher zapewnia doskonałe rozłożenie obciążenia między bufory (współczynnik zmienności < 0,3\% we wszystkich testach).
    
    \item \textbf{Wpływ liczby buforów} - Zwiększenie liczby buforów z 5 do 10 poprawiło równomierność rozłożenia obciążenia (0,1\% vs 0,2-0,3\%).
    
    \item \textbf{Optymalna konfiguracja} - Zrównoważona liczba producentów i konsumentów (1:1) przy dużej liczbie wątków (test 2) zapewnia najwyższą przepustowość.
    
    \item \textbf{Stabilność} - System działa stabilnie we wszystkich konfiguracjach, różnica między wyprodukowanymi a skonsumowanymi elementami jest minimalna (<0,3\%).
    
    \item \textbf{Synchronizacja} - Zastosowane mechanizmy synchronizacji (AtomicInteger, BlockingQueue) zapewniają bezpieczną współbieżność bez deadlocków.
\end{enumerate}

\section{Możliwe usprawnienia}

\begin{itemize}
    \item Implementacja dynamicznego dostosowania liczby wątków w czasie działania
    \item Priorytetyzacja buforów w zależności od obciążenia
    \item Monitoring wydajności w czasie rzeczywistym
    \item Optymalizacja strategii wyboru bufora przez producenta
    \item Dodanie mechanizmu obsługi przeciążenia systemu
\end{itemize}

\section{Podsumowanie}

Zaimplementowany system producent-konsument z tablicą dispatcher spełnia wszystkie założone cele. Wykazuje doskonałą skalowalność, efektywne równoważenie obciążenia oraz stabilność działania w różnych konfiguracjach. Przeprowadzone testy potwierdzają poprawność implementacji mechanizmów synchronizacji i możliwość wykorzystania systemu w aplikacjach wymagających wysokiej przepustowości.

\end{document}